\documentclass[a4paper, 10pt, english]{article}
\usepackage{multicol}
\textheight=26cm
\textwidth=17cm
\topmargin=-2cm
\oddsidemargin=-0.5cm

\title{Object Browser - Learning OOP}
\author{
        Guillermo Polito \\
        \and
        Guillermo Polito \\
        \and
        Guillermo Polito \\
}

\setlength{\columnsep}{25px}

\begin{document}
\maketitle


\begin{abstract}
This paper presents an alternative way to introduce students to the object oriented programming paradigm.  The objective is to reduce the necessary time to develop programs, taking advantage of the concepts we consider core of the paradigm: delegation, polymorfism and object composition.

In order to reach the proposed objective, we minimize the set of structural elements usually introduced at the beginning of an object oriented course-like classes and inheritance.  We built a development environment which allows to build programs following the object oriented principles and based on the concepts of object, message and reference.

This paper also introduces the Object Browser-the tool that gives support to this proposal- and analizes how each of it's features help in the learning of the paradigm.

Finally, we present a discussion about the results we obtained using this tool, proving it helps the student understand the object oriented paradigm and it contributes to the aquisition of programming best practices in a clear and solid manner since their first steps.s
\end{abstract}

\begin{multicols}{2}

\section{Introduction}

Why this is a problem\\
Your solution\\
Why your solution solves the problem\\
Experiments / Validation\\

\paragraph{Outline}
The remainder of this article is organized as follows.\cite{Gil:02}
Section~\ref{related work} gives account of previous work.
Our new and exciting results are described in Section~\ref{your solution}.
Finally, Section~\ref{conclusions} gives the conclusions.\\
 (Esto esta bueno? Guille)

\section{Problem}\label{problem}
Which is the problem?\\
Constraints\\
Tips (Pistas) of the Solution\\
Solution in 5 lines\\

\section{Your Solution}\label{your solution}
Larala

\section{Discussion}\label{Discussion}
In this section we describe the results.

\section{Related Work}\label{related work}
How do we position our work compared to other ones?

\section{Conclusions}\label{conclusions}
– What we did
– General Lessons
– Future Work

\bibliographystyle{abbrv}
\bibliography{simple}

\end{multicols}

\end{document}
This is never printed
